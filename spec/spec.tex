\documentclass{article}
\usepackage[italian]{babel}
\title{Relazione SecureDataContainer}
\author{Giovanni Solimeno}

\begin{document}
	\maketitle

	\newpage

	\section[Scelte progettuali]{Scelte progettuali}
	In generale, si è deciso di creare una classe contenitrice \texttt{Element<E>} per incapsulare il tipo generico, in modo da rendere più facili le operazioni sui permessi, e una classe \texttt{User}, rappresentante il singolo utente.

\subsection{\texttt{Element<E>}}
La classe \texttt{Element<E>} contiene tre campi privati, con relativi metodi setter/getter:
\begin{itemize}
\item Il campo \texttt{owner} è una stringa contenente il proprietario del dato.
Non è accessibile direttamente, ma è possibile controllare se un particolare utente è prioprietario del dato tramite il metodo \texttt{ownedBy(who)}.

\item Il campo \texttt{allowed} è una lista di stringhe, contenente gli utenti autorizzati ad accedere al dato (escluso il proprietario).
Non è possibile accedere direttamente alla lista, ma è possibile indicare che un utente deve essere autorizzato/disautorizzato tramite i rispettivi metodi \texttt{allowUser(other)} (che lancia \texttt{UserAlreadyAllowedException} se l'utente è già autorizzato) e \texttt{denyUser(other)} (che lancia \texttt{UserNotAllowedException} in caso l'utente non sia presente tra gli utenti autorizzati).

\item Il campo \texttt{el} è una riferimento ad una istanza del tipo generico E. È possibile accedervi tramite il metodo \texttt{getEl()}, che restituisce un riferimento \texttt{el}, mentre non è possibile cambiarne il valore.
\end{itemize}
Inoltre, è possibile controllare se un utente può accedere a un dato tramite il metodo \texttt{canBeAccessedBy(other)}, che restituisce true se e solo se other è il proprietario oppure è presente nella lista degli utenti autorizzati.\break
La classe sovrascrive il metodo \texttt{Object.equals(other)}, in modo da ritornare true se e solo se \texttt{other.owner.equals(this.owner)} è true e \break\texttt{other.getEl().equals(this.getEl())} restituisce true. \break Viene generata l'eccezione unchecked \texttt{NullPointerException} se \texttt{other} è nullo (la scelta di chiamare \texttt{equals} su \texttt{other.owner} e su \texttt{other.getEl()} è stata fatta in modo da lanciare in automatico l'eccezione se \texttt{other} è \texttt{null}).

\subsection{\texttt{User}}
La classe \texttt{User} contiene due campi privati:
\begin{itemize}
\item Il campo \texttt{userName} contiene il nome utente dell'utente, ed è possibile accedervi tramite il metodo \texttt{getUserName()}. Non è possibile in alcun modo modificarne il valore.

\item Il campo \texttt{userPass} contiene la password dell'utente, ed è possibile soltanto modificarla,  tramite il metodo \texttt{setUserPass( newPass )}, mentre non è possibile accedervi in alcun modo.

Inoltre, la classe implementa come meccanismo di login la sovrascrittura del metodo \texttt{Object.equals(other)}, che restituisce true se e solo se \texttt{other.getUserName().equals( this.getUserName )} e \newline
\texttt{other.userPass.equals (this.userPass)} (si è deciso di effettuare il confronto tramite i metodi/campi di other per lo stesso motivo di   \newline\texttt{Element<E>.equals()}), e implementa l'interfaccia \texttt{Comparable<T>}, in modo da ordinare gli utenti in base al nome (proprietà che viene usata nella classe \texttt{TreeMapSecureDataContainer}).
\end{itemize}

\section{Scelte specifiche}
Si è scelto di non imporre vincoli sulle classi/interfacce del tipo generico di \texttt{SecureDataContainer}, in modo da rendere l'utilizzo dell'interfaccia facile e non creare problemi di compatibilità, e di non criptare gli elementi salvati.
\subsection{ListSecureDataContainer}
Si è deciso di fornire le due implementazioni usando due metodologie diverse:
La prima (\texttt{ListSecureDataContainer}) si appoggia su due liste non ordinate, la prima contenente gli utenti registrati, mentre la seconda contenente i dati degli utenti.


\subsection{TreeMapSecureDataContainer}
La seconda (\texttt{TreeMapSecureDataContainer}) si basa su una TreeMap (ordinata tramite \texttt{User.compareTo()}), che associa ad ogni utente una lista con gli elementi da lui posseduti.
Nonostante ciò, la ricerca/rimozione/condivisione di un elemento non si avvale delle proprietà di un albero, in quanto bisogna scorrere ogni associazione utente/lista elementi, alla ricerca degli elementi condivisi (oltre a quelli posseduti, se non ci fosse bisogno di cercare tali elementi basterebbe controllare nella lista associata all'utente che ha richiesto l'operazione).
\end{document}
